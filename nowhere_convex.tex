\documentclass[12pt]{article}
\usepackage[english]{babel}
\usepackage[utf8]{inputenc}
\usepackage{inputenc}
\usepackage[a4paper, left=2.0cm, right=2.0cm, top=2.0cm, bottom=2.0cm, headsep=0.0cm]{geometry}
\usepackage{amssymb}
\usepackage{amsmath}
\usepackage{amsthm}
\usepackage{lscape}

\usepackage[ddmmyyyy,hhmmss]{datetime}
\renewcommand{\dateseparator}{.}

\usepackage{hyperref}

\newcommand{\R}{\mathbb{R}}
\newcommand{\N}{\mathbb{N}}

\newcommand{\dd}{\,\mathrm{d}}

\newcommand{\dprime}{{\prime\prime}}
\renewcommand{\leq}{\leqslant}
\renewcommand{\geq}{\geqslant}
\newcommand{\mapto}{\longmapsto}
\newcommand{\convto}[2]{\xrightarrow[#2]{#1}}

\newcommand{\ContClass}{\mathcal{C}}

\newcommand{\define}[1]{\textit{#1}}
\newcommand{\abs}[1]{\left| #1 \right|}
\newcommand{\paren}[1]{\! \left(#1 \right)}
\newcommand{\bracket}[1]{\! \left[#1 \right]}


\theoremstyle{plain}
\newtheorem{theorem}{Theorem}
\newtheorem{lemma}[theorem]{Lemma}
\newtheorem{corollary}[theorem]{Corollary}
\newtheorem{proposition}[theorem]{Proposition}

\theoremstyle{definition}
\newtheorem{definition}[theorem]{Definition}

\theoremstyle{remark}
\newtheorem{remark}[theorem]{Remark}
\newtheorem{example}[theorem]{Example}


\makeatletter
\def\blfootnote{\xdef\@thefnmark{}\@footnotetext}
\makeatother

\begin{document}
\noindent Piotr Idzik \hfill ver. \today\ \currenttime{}

\vspace*{1cm}
In this note we will construct (cf.\ formulas~\eqref{eq::definition_of_d_and_f}) an example of
a non-negative, strictly increasing, continuously differentiable function $d \colon \left[0, \infty \right) \to \left[0, \infty \right)$,
bounded from above by a convex function $f\colon\left[0, \infty \right) \to \left[0, \infty \right)$, $f\paren{0} = 0$,
such that the function $d$ is not convex on any interval $I \subseteq \left[0, \infty \right)$.
Such considerations are interesting from the point of view of~\cite[p. 167]{Levandosky1998}.
It is (implicitly) stated there, that if a decreasing function
$\widetilde{d} \geq 0$ is bounded from above by a convex function $\widetilde{f}$
vanishing at some point $b > 0$,
then the function $\widetilde{d}$ is convex on some interval arbitrary close to $b$.
After considering the mappings
\begin{equation*}
    d\paren{x} =  \widetilde{d}\paren{b-x}, \ f\paren{x} =  \widetilde{f}\paren{b-x} \quad \paren{x \in \bracket{0, b}}, \\
\end{equation*}
one sees that instead of decreasing function $\widetilde{d}$, we can consider increasing function $d$. 

We will use the following results
\begin{theorem}
    There exists a continuous, bounded and nowhere differentiable function $w \colon \R \to \R$.
\end{theorem}
Example of such function can be found in~\cite[Example 8, p. 38]{gelbaum1964counterexamples}.
 
\begin{theorem} \label{thm::convex_monotone_der}
    If the function $g$ is convex in an interval $\paren{a, b}$, then $g$ is continuous in $\paren{a, b}$.
    Moreover, $g^\prime$ exists except at most in a countable set and is monotone increasing.
\end{theorem}
The proof of above Theorem can be found in~\cite[Theorem 7.40, p. 120]{wheeden1977measure}.

\begin{theorem} \label{thm::monotone_diff_ae}
    Let $g \colon \paren{a, b} \to \R$ be monotone increasing.
    Then the function $g$ has a measurable, non-negative derivative $g^\prime$ almost everywhere in $\paren{a, b}$.
\end{theorem}
For the proof of the above we refer to~\cite[Theorem 7.21, page 111]{wheeden1977measure}.


Consider $\alpha \geq 0$ and let $w \colon \R \to \bracket{0, 1}$ be a continuous, nowhere differentiable function.
Define the functions $d, f \colon \left[0, \infty\right) \to \R$ as follows
\begin{align} \label{eq::definition_of_d_and_f}
    \begin{split}
        d\paren{x} & = \int_0^x s^\alpha w\paren{s} \dd s \quad \paren{x \geq 0}, \\
        f\paren{x} & = \int_0^x s^\alpha \dd s = \frac{x^{\alpha+1}}{\alpha+1} \quad \paren{x \geq 0}.
    \end{split}
\end{align}
 
\begin{remark} \label{remark::properties_of_f}
    The function $f$ is convex and $f\paren{0} = f^\prime\paren{0} = 0$.
\end{remark}
 %\begin{proof}
 % First consider the case $\alpha > 0$.
 % Note
 % \begin{equation*}
 %  f^\prime\paren{x} = x^\alpha, \, f^\dprime\paren{x} = \frac{1}{\alpha} x^{\alpha-1} \quad \paren{x\geq 0}.
 % \end{equation*}
 % Since $\alpha>0$, $f^\dprime\paren{x} > 0$ $\paren{x > 0}$ and $f^\dprime\paren{0} = 0$, $f$ is convex.
 %
 % If $\alpha = 0$,then $f\paren{x} = x$ $\paren{x \geq 0}$.
 %\end{proof}

\begin{remark} \label{remark::estimate}
    The following inequality holds true 
    \begin{equation*}
        0 \leq d\paren{x} \leq f\paren{x} \quad \paren{x \geq 0}.
     \end{equation*}
\end{remark}
\begin{proof}
    Observe that
    \begin{equation*}
        0 \leq d^\prime\paren{s} = s^\alpha w\paren{s} \leq s^\alpha = f^\prime\paren{s} \quad \paren{s > 0}.
    \end{equation*}
    Integrating above inequality over the set $\bracket{0, x}$ gives the claim.
\end{proof}
 
\begin{proposition} \label{prop::d_is_C1}
    The function $d$ is continuously differentiable and if $\alpha>1$ then $d^\dprime\paren{0} = 0$ (\footnote{here we consider $d^\dprime\paren{0}$ as the limit $\lim_{h \to 0^+} \frac{d^\prime\paren{h} - d^\prime\paren{0}}{h}$.}).
\end{proposition}
\begin{proof}
    Observe that $d^\prime\paren{x} = x^\alpha w\paren{x}$ $\paren{x \in \left[0, \infty\right)}$.
    Moreover, we have
    \begin{equation*}
        \frac{d^\prime\paren{h} - d^\prime\paren{0}}{h} = \frac{h^\alpha w\paren{\alpha}}{h} \convto{}{h \to 0^+} 0,
    \end{equation*}
    for all $\alpha > 1$.
\end{proof}

 
\begin{proposition} \label{prop::d_prime_is_not_differentiable}
    The function $d^\prime$ is not differentiable for $x>0$.
 \end{proposition}
\begin{proof}
    Suppose the contrary and consider the quotient
    \begin{align} \label{eq::prop::d_prime_is_not_differentiable_quotient}
      \begin{split}
          \frac{d^\prime\paren{x+h} - d^\prime\paren{x}}{h} & = \frac{\paren{x+h}^\alpha w\paren{x+h} - x^\alpha w\paren{x}}{h} \\
                                                            & = \frac{\paren{x+h}^\alpha w\paren{x+h} - \paren{x+h}^\alpha w\paren{x} + \paren{x+h}^\alpha w\paren{x} - x^\alpha w\paren{x}}{h} \\
                                                            & = \paren{x+h}^\alpha \frac{w\paren{x+h} - w\paren{x}}{h} + w\paren{x} \frac{\paren{x+h}^\alpha - x^\alpha}{h}
      \end{split}
    \end{align}

    By our assumption the quotient of the left hand side of~\eqref{eq::prop::d_prime_is_not_differentiable_quotient} converges to $d^\dprime\paren{x}$ as $h \to 0$.
    By the continuity of the function $w$ and and differentiability of the function $\paren{0, \infty} \ni s \mapto s^\alpha \in \paren{0, \infty}$, we get that the limit $\lim_{h \to 0} \frac{w\paren{x+h}-w\paren{x}}{h}$ exists.
    Therefore the function $w$ is differentiable at point $x$.
    Contradiction.
\end{proof}

\begin{proposition} \label{prop::d_is_increasing}
    The function $d$ is strictly increasing.
\end{proposition}
\begin{proof}
    Clearly $d^\prime\paren{x} =x^\alpha w\paren{x} \geq 0$ $\paren{x \geq 0}$, therefore
    \begin{equation*} %\label{eq::d_is_weakly_increasing}
        d\paren{a} - d\paren{b} = \int^a_b d^\prime\paren{s} \dd s \geq 0 \quad \paren{a \geq b \geq 0}.
    \end{equation*}

    Suppose that there exist some points $a > b \geq 0$, such that
    \begin{equation*}
        0 = d\paren{a} - d\paren{b} = \int^a_b d^\prime\paren{s} \dd s.
    \end{equation*}
    Since $d^\prime \geq 0$, we have that $d^\prime\paren{s} = 0$ for almost all $s \in \left[a, b\right]$.
    By the continuity of the function $d^\prime$, we have that $d^\prime\paren{s} = 0$ $\paren{s \in  \left[a, b\right]}$.
    Hence, the function $d^\prime$ is differentiable on the interval $\paren{a, b}$, which gives a contradiction (cf.\ Proposition~\ref{prop::d_prime_is_not_differentiable}).
\end{proof}

\begin{proposition} \label{d_is_non_convex_on_any_interval}
    The function $d$ is not convex on any of the intervals $I \subseteq \left[0, \infty \right)$.
\end{proposition}
\begin{proof}
    Suppose that there exists an interval $I \subseteq \left[0, \infty \right)$ such that the function $d_{|I}$ is convex.
    Theorem~\ref{thm::convex_monotone_der} implies that the function $d^\prime$ is monotone increasing.
    Therefore, by Theorem~\ref{thm::convex_monotone_der} the function $d^\prime$ is differentiable almost everywhere (with respect to Leagues measure) on $I$.
    This gives the contradiction together with Proposition~\ref{prop::d_prime_is_not_differentiable}.
\end{proof}

\begin{remark}
    Let the functions $d$ and $f$ be as above (cf.\ formula~\eqref{eq::definition_of_d_and_f}) and let $b>0$.
    Consider functions $\widetilde{d}, \widetilde{f} \colon \bracket{0, b} \to \left[0, \infty \right)$ defined as follows:
    \begin{align*}
        \widetilde{d}\paren{x} = d\paren{b-x} \quad \paren{x \in \bracket{0, b}}, \\
        \widetilde{f}\paren{x} = f\paren{b-x} \quad \paren{x \in \bracket{0, b}}.
    \end{align*}
    Then
    \begin{itemize}
        \item $0 \leq \widetilde{d}\paren{x} \leq \widetilde{f}\paren{x}$ for all $x \in \bracket{0, b}$ (cf.\ Remark~\ref{remark::estimate}),
        \item the function $\widetilde{f}$ is convex on the interval $\bracket{0, b}$ and $\widetilde{f}\paren{0} = \widetilde{f}^\prime\paren{b} = 0$ (cf.\ Remark~\ref{remark::properties_of_f}),
        \item the function $\widetilde{d}$ is of the class $\ContClass^1$ on $\bracket{0, b}$ (cf.\ Proposition~\ref{prop::d_is_C1}),
        \item the function $\widetilde{d}$ is strictly decreasing on the interval $\bracket{0, b}$ (cf.\ Proposition~\ref{prop::d_is_increasing}),
        \item the function $\widetilde{d}$ is not convex on any interval $I \subseteq \bracket{0, b}$ (cf.\ Proposition~\ref{d_is_non_convex_on_any_interval}).
    \end{itemize} 
\end{remark}

\bibliography{bib_data}{}
\bibliographystyle{plain}
\end{document}
 